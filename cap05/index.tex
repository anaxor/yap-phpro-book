\chapter{Securitatea aplicaţiilor web}
\vskip -25pt
\textit{Intro}
\vskip 5em

% Despre vectori de atac, ce înseamnă securitate, accent pe securitatea din perspectiva programatorului.

\section{Ce înseamnă securitate}
Securitatea este un subiect dificil deoarece nu există o reţetă
universal valabilă pentru a crea ceva sigur.

Dar de ce nu există
o astfel de armă secretă?
Pentru a înţelege asta, trebuie să înţelegem natura conceptului de
\textit{a fi sigur}. 

Pentru noi ca programatori, ne interesează în special securitatea
aplicaţiei. Însă aplicaţia noastră nu este de sine stătătoare: ea
se află pe un server care are un sistem de operare, care este conectat
la Internet, şi deseori la o reţea locală centrului de date în
care se află acest server, iar la acest server unele persoane
din cadrul firmei ce deţine centrul de date pot avea acces fizic.

Şi cum ceva ori este sigur, ori nu este, 
%TODO finish me
\section{Securitatea la nivelul aplicaţiei}
%TODO write me
verificarea inputului, POST forms don't make your app secure,
security by obscurity

includerea fişierelor, definirea unei constante "guard",
punerea fişierelor în afara \texttt{htdocs/}

XSS şi CSRF

%TODO implicaţiile de securitate "http is stateless"

%TODO exerciţiu: Al doilea exerciţiu de hacking
TODO: exerciţii de hacking. O infrastructură exactă răm\^ane să fie inventată
pentru scenarii mai complexe. Cel mai probabil cursanţii vor primi un
{\glqq}private key{\grqq}, a.\^i. toate atacurile (simulative, desigur, nu reale)
vor putea fi traced back.

TODO: exerciţii de securizare a scripturilor trecute.

TODO: ne trebuie o evaluare serioasă a vectorilor de atac dacă oferim
{\glqq}attack gates{\grqq}. Cu siguranţă vom verifica IP-ul, şi doar către
IP-ul care foloseşte private key se vor trimite atacurile (i.e. cursantul
va folosi serverul phpro pentru a-şi ataca propriul server -- proper NAT has to be
in place on his side).