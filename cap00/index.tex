\setcounter{page}{1}
\fancyfoot[CO,C] {\thepage}
\phantomsection
\addcontentsline{toc}{chapter}{Introducere}
\chapter*{Introducere}

\vskip -25pt

\textit{Acest capitol conține informații foarte importante despre cum să
studiezi, unde să ceri ajutor, cum să raportezi greșeli şi în general cum să
profiţi la maxim de materialul prezentat. Recomand citirea sa cu atenție.}


\vskip 5em

În ultimii ani, utilizarea Internetului a crescut rapid. Numărul de dispozitive
conectate la Internet creşte în mod exponenţial de la an la an, iar
web-ul\footnote
{World Wide Web} a devenit scena principală. Web-ul nu mai este static demult,
avem rețele sociale,\footnote{hi5, facebook, lastFM, delicious}
feed-uri,\footnote{RSS, Atom} bloguri și microbloguri,\footnote{twitter}
ș.a.m.d. Observăm că mai toate activitățile noastre informaționale s-au mutat 
pe web.

Însă aceste activități devin din ce în ce mai complexe, la fel ca și
aplicațiile\footnote{Termenul \glqq aplicație\grqq nu este tocmai corect, în sensul
tradițional, însă autorul consideră că limba este ceva maleabil, care se
schimbă în timp în funcție de nevoi.} care le susțin.
Iar aceste aplicații trebuie dezvoltate de cineva -- de programatori.

PHP este unul dintre cele mai folosite limbaje pentru crearea de aplicații
web dinamice. Succesul său se datorează în special simplității sale, însă
acest lucru e cu două tăișuri: pe de o parte este ușor accesibil, pe de
cealaltă parte poți face foarte ușor greșeli majore.

\phantomsection
\addcontentsline{toc}{section}{Scopul acestei cărţi}
\section*{Scopul acestei cărţi}

Pentru a scrie aplicații PHP bune\footnote{performante, sigure, complexe, mentenabile}
este cerut simțul critic al programatorului,
însă există o mare problemă: începătorul care alege PHP ca
primul său limbaj de programare nu are un simț critic dezvoltat sau gândirea analitică
necesare în programare. În combinație cu accesibilitatea \textit{aparentă}
a limbajului PHP, acest lucru se dovedește fatal pe \textit{termen lung}.

Scopul acestei cărţi este să-ți dezvolte gândirea autonomă, productivă,
critică, cât și capacitatea de analiză și sinteză, capacități atât de vitale
în programare.
%-----------------------------------------------------------------------------

Lucrarea de față nu este nici pe departe completă -- nici nu vrea să fie.
Tot ceea ce vrea este să-ți ofere o fundație bună de start. Acest lucru se 
face punând accent pe \textit{terminologie} și pe explicarea
\textit{modului de funcționare} a noţiunilor şi tehnologiilor prezentate.

După cum probabil intuieşti deja, scopul acestei cărţi este să
te susţină să devii bun în perspectivă, fiind orientată spre
viitor, pe termen lung.

\phantomsection
\addcontentsline{toc}{section}{De ce am nevoie? Premize}
\section*{De ce am nevoie? Premize}
Această lucrare pleacă de la premiza că ştii deja (X)HTML,\footnote{În
ciuda credinţelor populare, HTML nu este un limbaj de programare.} eventual şi CSS, dar
acesta din urmă nu este necesar pentru înţelegerea lucrurilor prezentate
sau pentru învăţarea PHP. Ar trebui
studiat oricum, căci fără el nu este posibil \textsl{design}-ul de
\textsl{website}-uri aspectuoase.
Acolo unde va fi nevoie, vor fi prezentate şi noţiunile JavaScript\footnote{Un
limbaj de programare pentru programarea clientului, a \textsl{browser}-ului,
în contrast cu PHP, cu care se programează \glqq serverul\grqq.
În ciuda credinţelor din folclor, JavaScript şi Java sunt limbaje complet
diferite şi de sine stătătoare.} necesare.

Un lucru important de care ai nevoie este răbdare. Citește cu
atenție și încearcă să înțelegi tot, căci informația este comprimată
și uneori pare să nu aibă nicio aplicabilitate practică, însă e doar o
iluzie -- tot ce scrie în acest ghid scurt este important. Nu uita că
\textit{mă rezum doar la fundație, la cunoștințele de bază}.

Aşa cum spune şi coperta acestei cărţi, plec de la premiza că
\textit{vrei să devii profesionist în PHP}. Dacă \textit{nu} asta este intenția ta,
atunci lucrarea de faţă nu este potrivită pentru tine.
În particular, în timp ce urmezi această carte pentru prima oară, nu o face
pentru a-ți rezolva problema ta imediată de care tocmai te-ai lovit, ci
încearcă să înțelegi noțiunile expuse și să rezolvi exercițiile prezentate.
Vei vedea că asta îți salvează mult timp și frustrare pe \textit{termen lung},
şi că rezolvarea eventualei probleme imediate de care te-ai lovit este de fapt
marginală carierei tale de \textit{programator profesionist}.

\attention{Ca un viitor profesionist ce eşti, citeşte cu atenţie acest material, şi
încearcă să înţelegi nu numai conceptele de care te loveşti, ci şi
implicaţiile lor. Analizează-le, atât pe el însele, cât şi în relaţie
cu celelalte concepte introduse. Cu cât sintetizezi mai mult atunci
când întâlneşti ceva nou, cu atât vei ajunge să \textit{jonglezi} cu noţiunile
învăţate mai rapid, lucru care îţi va permite să fii inovativ.

De exemplu, în primul capitol vei învăţa despre reţelistică. Întreabă-te
pe parcursul întregii cărţi ce efecte au limitările HTTP asupra posibilităţilor
sau asupra securităţii.}

\phantomsection
\addcontentsline{toc}{section}{Convenţii folosite}
\section*{Convenţii folosite}
Pentru a-ţi face lectura cât mai plăcută, cartea de faţă respectă anumite
convenţii, atât de natură tipografică, cât şi inerente comunităţii din jurul
lucrării.

În primul rând, pe prima pagină se află un \textsl{link} către pagina de
start a proiectului. Această pagină va fi numită mereu \textit{pagina \phpro}.
Următoarea secţiune îţi va descrie despre ce este vorba în detaliu.

În al doilea rând, când spun wikipedia, mă refer la versiunea originală
în engleză a site-ului \url{http://en.wikipedia.org/}. În carte nu voi face referire
la niciun \textsl{site} în română.


{\glqq}Pagina PHP{\grqq} şi {\glqq}manualul PHP{\grqq} se referă la paginile oficiale \url{http://php.net/}
şi respectiv \url{http://www.php.net/manual/en/}. Demn de menţionat este
că nu voi folosi decât \textit{site}-urile oficiale ale produselor despre
care vorbesc.

\attention{Urmarea link-urilor, în special cele către wikipedia şi către
manualul PHP, \textit{nu} este
opţională. Informaţiile prezentate în acele pagini \textit{fac parte} din cartea
de faţă.}

Cuvintele importante sunt scrise \textit{cursiv}, termenii şi noţiunile
importante sunt scrise \textsl{înclinat}, iar cuvintele care fac referire
la nume de funcţii, comenzi sau instrucţiuni care trebuie introduse
într-un fişier sau ca comandă,
sau cuvinte cheie specifice unui anumit limbaj
sunt scrise cu font \texttt{neproporţional}\footnote{\href{http://en.wikipedia.org/wiki/Monospaced_font}{monospace}}.

Apăsările de taste sunt scrise în chenare, astfel \keystroke{ENTER} înseamnă
să apeşi enter o dată, iar \keystroke{CTRL+F} înseamnă să apeşi tasta \keystroke{CTRL}
şi în timp ce o ţii apăsată, să apeşi \keystroke{F}.

Codul sursă, de obicei în PHP, va arăta în felul următor:

\vskip 3em
\begin{lstlisting}[caption={Convenţie listare}]
<?php
phpinfo();
\end{lstlisting}

Numerele de pe marginea stângă reprezintă numerele liniilor de cod corespunzătoare
şi nu trebuie scrise. Ele deservesc unei mai uşoare identificări în explicaţiile
din text.

Codul sursă va conţine şi caracterul ’. Acesta trebuie văzute ca
apostrof. Altfel spus, copierea şi lipirea codului sursă prezentat
nu va funcţiona pur şi simplu. Acesta trebuie scris de tine însuţi, deoarece
caracterul respectiv este (intenţionat) greşit. Procedez astfel pentru
a împiedica copierea codului sursă. Acesta trebuie înţeles.

Atunci când introduc termeni noi, încerc să ofer şi traducerea în
engleză, în paranteză.
Exemplu: \begin{quote}
         	Un astfel de atac se numește man in the middle, deoarece atacatorul se află la mijloc, între
cele două capete (en. \textsl{endpoints}).
         \end{quote}
iar la sfârşitul cărţii poţi găsi o referinţă a tuturor acestor termeni.

Există trei feluri în care marchez paragrafe:

\attention{Paragrafele care te atenţionează asupra unui lucru important
sunt marcate ca atare, ca cel de faţă.}

\good{În unele locuri vreau să-ţi atrag atenţia asupra unei practici de programare bune.}

\bad{În acelaşi timp, câteodată îţi atrag atenţia asupra unor lucruri care, deşi sunt posibile, nu ar trebui făcute.}

Exerciţiile sunt marcate cu un creion, iar numărul de steluţe reprezintă dificultatea lor, între 0 (nicio steluţă)
şi 3 (trei steluţe). Exemplu:
\begin{Exercise*}[title={Exerciţiu de dificultate 1},difficulty=1]
	Eu sunt un enunţ.
\end{Exercise*}

Exerciţiile de dificultate zero necesită doar înţelegerea exemplelor şi
explicaţiilor imediat anterioare şi mici modificări sau adăugiri.
Cu cât solicitarea inteligenţei şi a capacităţii de sinteză a cursantului
creşte, cu atât creşte şi numărul steluţelor.

La evaluarea dificultăţii exerciţiilor procedez în
felul următor: în primul rând, plec de la premiza că
cursantul ştie toate noţiunile din capitolul anterior,
chiar dacă nu le-a sintetizat pe toate. În acelaşi timp,
plec de la premiza că restul capitolelor trecute
au fost bine sintetizate.

\attention{Dacă trebuie să sari cu mai mult de un
capitol înapoi pentru a revizui ceva, atunci este
un indiciu că nu ai trecut prin toate stadiile
de studiu în mod consecvent. Secţiunea următoare
îţi va explica care sunt aceste stadii.}

%La începutul unui
%capitol plec de la premiza că cursantul
%a înţeles deja noţiunile capitolului anterior
%jonglează deja fără probleme
%cu noţiunile acoperite de capitolele anterioare. Altfel spus, nu te poţi
%aştepta ca exerciţiile dintr-un nou capitol să fie uşoare, dacă
%nu ai trecut prin toate stadiile unui studiu corect. Următoarea
%secţiune îţi va prezenta aceste stadii şi care sunt beneficiile lor.

Dacă observi încălcări ale acestor convenţii, te rog să le
raportezi pe pagina de greşeli a \phpro.
\phantomsection
\addcontentsline{toc}{section}{Cum să înveţi eficient programare}
\section*{Cum să înveţi eficient programare}

%TODO at chapter 6: remove this
Momentan cartea de faţă nu acoperă încă materia aşa cum aş vrea -- nu este completă.
Însă subiectele abordate sunt acoperite complet, cel puţin la nivel conceptual.

Cartea în sine nu este gândită pentru a fi folosită singură, ci
în paralel cu comunitatea \phpro. În particular, unele exerciţii
chiar nu sunt gândite pentru a fi rezolvate de cititor singur,
ci cu susţinerea tutorilor de pe \phpro.

Pe {\phpro} găseşti şi ajutor sub formă de idei şi indicii pentru rezolvarea
exerciţiilor.

Învaţă \textit{terminologia}, înţelege-o şi foloseşte-o. Dacă setul de scule de programare folosite
este lancea ta de programator, atunci terminologia este vârful lancei.
Care este diferenţa dintre un toiag tocit, şi o lance fără vârf?
Exact, nici una. Nu te apuca să foloşesti termeni pe care nu-i înţelegi,
ci documentează-te înainte. Cu o lance ascuţită:
\begin{itemize}
	\item te vei putea înţelege mai uşor cu alţi programatori; tu îi vei înţelege pe ei, şi ei pe tine
	\item pe măsură ce termenii înţeleşi de tine devin mai complecşi,
		  vei putea acumula cunoştinţe din ce în ce mai complexe bazate pe cele anterioare,
		  în ritm exponential. La început ţi se va pare frustrant, însă dacă vrei să devii bun,
		  oricum va trebui să înveţi termenii odată şi-odată. Deci de ce să nu faci totul ca
		  la carte de la bun început?
	\item un programator profesionist ştie mai mult de un singur limbaj de programare; ai fi
		  uimit dacă ai afla câţi termeni şi câte concepte sunt comune multor limbaje. Dacă
		  ştii terminologia, chiar dacă ai învăţat-o în (cu) PHP, vei putea trece la un nou
		  limbaj cu mult mai puţine eforturi. Primul limbaj (învăţat corect) este cel mai greu,
		  apoi ţi se va pare floare la ureche
\end{itemize}


Urmează \href{http://en.wikipedia.org/wiki/Hyperlink}{link}-urile în timp ce
studiezi;
acestă carte nu este şi nu va fi niciodată {\glqq}completă{\grqq} -- se pleacă de la premiza
că citeşti şi înţelegi ce se află la acele link-uri \textit{înainte} de a trece
mai departe.

Notele de la subsol sunt importante; dacă acestea introduc termeni neexplicaţi
anterior sau în imediata vecinătate, atunci trebuie reţinute şi făcute
legături atunci când termenii respectivi sunt introduşi pentru prima oară.

Plec de la premiza că cititorul meu are un anumit nivel de inteligenţă.
Asta nu înseamnă că nu iau în serios orice nelămurire. Însă mă aştept
ca noţiunile prezentate să fie citite cel puţin, şi apoi înţelese.
Nu are rost să citeşti o carte dacă ... nu o citeşti cu trup şi suflet.
Atunci când ai o problemă, ia-o gradual, netrecând la următorul stadiu
până nu îl îndeplineşti pe cel anterior.

\attention{Stadiile\footnotemark sunt: citire, înţelegere, sinteză, imaginaţie (jonglarea
cu noţiunile), inovaţie.}
\footnotetext{Noţiunea de
\textit{stadiu de învăţare} este extinderea autorului a sistemului japonez
shu-ha-ri (en. \textit{retain-detach-transcend}, jp. \jptext{守 破 離}). Detalii pe
\url{http://www.makigami.info/cms/japanese-learning-system-japan-36}.}

A sintetiza înseamnă a face legături cu toate celelalte noţiuni
deja învăţate. De exemplu vei învăţa ce înseamnă un \textsl{array}, iar peste
câteva capitole vei face cunoştinţă cu obiecte. Dacă vei sintetiza
cum trebuie, îţi vei da seama singur că este foarte posibil
să ai un array de obiecte.

A jongla cu noţiunile are ca efect practic faptul că cititorul ştie
să pună în practică şi să combine lucrurile învăţate de ca şi cum
acele noţiuni ar fi fost inventate de el.

\attention{Îţi poţi uşura procesul de sinteză asimilând terminologia
încă din momentul introducerii ei.}

Această sinteză e foarte importantă, şi de fapt, o faci de când erai
copil. De exemplu, ai văzut-o pe mama ta tăind legumele cu cuţitul.
Mai târziu, la joacă, ai avut nevoie să tai o aţă, şi nu aveai decât
un cuţit în apropiere. Ţi-ai dat seama că poţi tăia aţa cu cuţitul,
deşi nu este o legumă. Altfel spus, ai sintetizat scopul uneltei
{\glqq}cuţit{\grqq}: să taie ceva.

Lucrarea de faţă explică foarte bine noţiunile, de la zero, însă
sinteza îţi este lăsată ţie. Motivaţia mea de a proceda aşa este următoarea:
după cum sugerează subtitlul cărţii -- \textit{Pentru începătorii în programare şi în PHP care vor să devină profesionişti} -- scopul meu e să te îndrum pe calea profesionalismului.
Pe de cealaltă parte, sunt un darwinist convins, şi dacă nu reuşeşti
nici să devii profesionist, nici să vezi utilitatea acestei cărţi, atunci
e mai bine aşa. Ultimul lucru pe care îl vreau este să te susţin
să devii ceva în care nu ai avea succes.

%TODO uncomment at chap 6
Capacitatea de sinteză pe care o vei fi având la sfârşitul cărţii
mai are încă un efect pozitiv asupra viitorului profesionist din tine:
în programare, vei fi confruntat cu nevoia de a reutiliza codul pe care-l scrii, astfel
încât să nu fii nevoit să rescrii acelaşi cod iar şi iar, doar pentru
că trebuie să-l personalizezi puţin. Însă pentru a putea face
codul atât de flexibil încât să-l poţi adapta cu uşurinţă, trebuie
să prevezi cazuri {\glqq}imprevizibile{\grqq}; altfel spus, să te gândeşti
la imposibil.

\good{Nu copia pur şi simplu exemplele din carte, pentru că rişti
să te trezeşti un moment dat că nu eşti în stare
să scrii ceva de unul singur. În schimb citeşte cu atenţie codul
şi explicaţiile de dinaintea şi după el, apoi \textit{închide cartea} şi scrie totul
din minte, argumentându-ţi (pe baza explicaţiilor pe care le-ai citit)
de ce faci un lucru într-un anumit fel, sau de ce îl faci de fapt.}

Ştiu că este mai uşor să copiezi, dar vor veni vremuri când va
trebui să inventezi singur un script. Deci obişnuieşte-te de
pe acum să scrii singur, şi de ce nu, să faci greşeli. Atunci
când faci o greşeală şi PHP îţi spune asta, citeşte cu atenţie
mesajul de eroare, apoi corectează-ţi codul, şi ţine minte
pentru fiecare fel de greşeală ce eroare generează, pentru ca
în viitor să poţi identifica mai rapid greşelile pe care le faci
pe baza mesajelor de eroare pe care ţi le arată PHP.

\attention{Această \textit{putere de imaginaţie}, în combinaţie
cu \textit{capacitatea ta de analiză şi sinteză}, şi pe o fundaţie solidă
a \textit{înţelegerii conceptelor şi termenilor} cu care intri în contact,
sunt cheia succesului garantat.}

\phantomsection
\addcontentsline{toc}{subsection}{Comunitatea}
\subsection*{Comunitatea}
\textit{Dezvoltare web cu PHP -- Pentru începătorii în programare şi în PHP care vor să devină profesionişti}
nu este pur şi simplu o carte, ci o comunitate şi o serie de servicii pe care
această comunitate le oferă. Cartea de faţă constituie doar scheletul, fundaţia
studiului. Pentru a beneficia deci de aceste servicii, cititorul cărţii
trebuie să fie şi cursant în cadrul comunităţii.

Pagina {\phpro} este pagina de start a comunităţii. Printre serviciile oferite se numără:
\begin{itemize}
	\item verificarea soluţiilor exerciţiilor şi oferirea de indicii acolo unde cursantul s-a blocat, individual,
pentru fiecare cursant în parte, exact acolo unde are nevoie
	\item clarificarea nelămuririlor pe care cursantul le are în urma citirii explicaţiilor
	\item articole care întregesc conceptele prezentate în carte; excursuri
	\item garanţia că cursanţii\footnote{În special cei care au reuşit
să ofere soluţii la primele trei exerciţii din capitolul 2, eventual cu susţinerea
tutorilor} au într-adevăr potenţialul de a deveni profesionişti
	\item servicii care sunt folosite în viaţa reală a unui programator
\end{itemize}

Ceea ce comunitatea {\phpro} nu este, este un loc unde poţi primi ajutor la
problemele de care te-ai lovit pe cont propriu. Altfel spus, comunitatea
noastră este strict una de studiu.

\phantomsection
\addcontentsline{toc}{subsection}{Exerciţiile}
\subsection*{Exerciţiile}

Exerciţiile sunt parte integrantă a studiului. Scopul exerciţiilor nu
este numai de a te testa, ci şi de a te învăţa lucruri noi. De fapt,
unele exerciţii au menirea exclusivă de a te învăţa ceva.

Indiferent de menirea fiecărui exerciţiu, poţi apela la comunitatea
{\phpro} pentru susţinere, sfaturi şi indicii la exerciţii. În fapt,
chiar va trebui să o faci la unele exerciţii -- vei avea nevoie de asta.

Desprinzăndu-te de comunitatea \phpro, rişti să studiezi ceva de unul
singur şi să ai impresia că ai înţeles totul corect, însă lucrurile învăţate
se pot aşterne greşit în mintea ta, şi la un moment dat te vei lovi
tu însuţi de probleme din cauza asta.

Având însă permanent, la fiecare exerciţiu, un tutore lângă tine care te
îndrumă, şansele ca un concept de programare să fie înţeles şi aplicat
greşit scad considerabil.

Unele exerciţii vor fi direct legate de comunitate şi de serviciile pe care
aceasta le oferă. În capitolul patru de exemplu, exerciţiile îţi vor 
cere să formezi echipe cu alţi cursanţi, şi să concurezi împotriva altor echipe, folosind
scule de programare aşa cum sunt folosite în viaţa reală a unui programator,
precum un \textsl{bug tracker} sau un \textsl{revision control system}.

Însă pentru a primi acces la aceste servicii pe care comunitatea
{\phpro} le oferă gratis, trebuie să rezolvi toate exerciţiile anterioare
sub tutela comunităţii, dovedind astfel că ai potenţialul unui
programator bun.

\phantomsection
\addcontentsline{toc}{section}{Cum pot ajuta?}
\section*{Cum pot ajuta?}
Atât programatorii experimentaţi, cât şi începătorii, pot ajuta,
iar ajutorul lor este apreciat în egală măsură.

Punctul de întâlnire pentru toţi este \phpro, unde poţi
găsi îndrumare despre ce poţi face, sau unde poţi raporta
ce ai de raportat.

De la cititorii avansați mă aștept la critică constructivă, sfaturi sau idei.
\textsl{Feedback}-ul mă bucură, însă vreau să atrag atenția asupra unui lucru:
există situații în care, atunci când trebuie să explici ceva, trebuie să
faci compromisuri între corectitudinea tehnică și ușurința cu care noțiunile
pot fi acumulate de cititor (en. \textsl{the learning curve}), iar cu compromisurile
suntem obișnuiți din programare. Așa se face că pe alocuri ofer explicații
nu tocmai corecte, care sunt corectate apoi. Asigură-te că ai citit tot conținutul
relevant (şi mai ales notițele de la subsol) înainte de a raporta o greșeală --
cel mai probabil explicațiile sau definițiile sunt reluate și șlefuite undeva.

În privința calității cărţii, există trei mari probleme:
\begin{itemize}
\item Nu cred în cacofonii. Consider că propria imaginație e singura vinovată
dacă {\glqq}vezi{\grqq} alte lucruri când citești. Ca atare, refuz să le corectez.
Ba mai rău, corectarea lor prin folosirea virgulei sau reformulări mai mult
ar îngreuna inteligibilitatea.
\item Folosesc xenisme. Resursele în limba engleză sunt cele mai acurate și
cele mai actuale, din acest motiv nu încerc să evit folosirea lor.
Asta îți va permite, pe \textit{termen lung}, să te poți ajuta singur. Pentru a
articula un xenism pun cratimă, și apoi particula specifică. De exemplu
\textsl{web}-ul; însă: \textit{Internetul} -- deoarece cuvântul internet există în
limba română.
\item Este foarte posibil să întâlnești formulări ciudate, cu ordinea
cuvintelor inversată, și altfel de greșeli similare. Cauza acestui lucru este că
90\% din timp vorbesc germana, lucru care-și lasă amprenta.
Corecturile sunt binevenite.
\end{itemize}

\phantomsection
\addcontentsline{toc}{section}{O privire de ansamblu a capitolelor}
\section*{O privire de ansamblu a capitolelor}
%TODO aici vine roadmap-ul

\begin{enumerate}
\item reţelistică
\item controlul fluxului de execuţie şi de date, prelucrări simple de formulare, tipuri de date
\item funcţii, modularizare şi separare a logicii de vizualizare. securitate
\item baze de date - mysql, sqlite; lucrul în echipă (subversion, bug tracker); debugging
\item OOP, concepte generale, câteva patterns (helper, strategy, factory, singleton),
documentare, testare (TDD), SPL
\item ajax, json, servicii (REST, SOAP, XML-RPC), XML, PDO şi alte delicii, şi ca
{\glqq}ultima frontiera{\grqq}: php internals.
\end{enumerate}

%TODO rephrase
Capitolele 3,4 şi 5 vor conţine câte un proiect, făcut pas cu pas, cu explicaţii sistematice
despre toate noţiunile introduse, iar la sfârşitul fiecărui capitol sunt propuse îmbunătăţiri
ale acelui proiect şi eventual alte proiecte.

În capitolul 4 se va oferi şi posibilitatea programării de proiecte în condiţii reale, în
echipe, care vor concura.
Participanţii vor fi tutelaţi de cei care au absolvit deja acel capitol.