\chapter{Baze de date şi lucrul în echipă}
\vskip -25pt
\textit{În acest capitol vei învăţa fundamentele bazelor de date şi lucrul într-o echipă de programatori,
cu toate uneltele necesare şi unele \textit{soft skills} de care are nevoie un programator.}
\vskip 5em

\section{Istoria unui proiect}

%TODO vorbeşte despre RCS-uri, despre cum se foloseşte unul, despre github
%exerciţiu: învaţă să lucrezi cu git şi github, point to progit.org, crează-ţi un repo cu acelaşi nume ca username-ul tău şi exersează acolo
%exerciţiu: fork yap-phpro-book, cum să trimiţi issues, patches, pull requests

\section{Baze de date relaţionale}
%TODO fill me

% Articole, de citit cu atenţie:
% 
% http://en.wikipedia.org/wiki/Relational_database
% 
% http://en.wikipedia.org/wiki/Database_normalization - şi articolele despre
% 1NF, 2NF şi 3NF - doar atât e suficient pentru orice programator
% 
% http://dev.mysql.com/doc/refman/5.6/en/tutorial.html - e destul de suficient
% pentru un început
% 
% după ce ai învăţat să foloseşti clientul mysql pentru a te conecta la daemonul
% mysqld, poţi instala şi phpmyadmin şi îl folosi pe el pentru a interacţiona
% cu serverul: http://www.phpmyadmin.net/home_page/index.php
% 

\section{Lucrul în echipă}
% Pentru lucrul în echipă:
% 
% - învaţă să lucrezi cu git: http://www.ralfebert.de/tutorials/git/
% 
% 
% - apelează la echipa YAP pentru următorii paşi (IRC: irc.freenode.net / #yet-another-project)
%TODO exerciţiu: găseşte 1-2 cursanţi pe IRC de acelaşi nivel, formaţi o echipă şi creaţi un proiect mai amplu
%tutorul va fi mediatorul vostru, dar nu se va băga peste ideile voastre (cel puţin nu în mod abuziv, ci doar ca să vă prevină greşelile sau să vi le îmbunătăţească)
%dacă e nevoie, fă schimb de numere de telefon sau alte mijloace de comunicare prin voce (ex: skype) cu membrii echipei tale, pentru a înlesni comunicarea
